\newcommand{\makeBox}[2] {
  \newsavebox{#1}
  \begin{lrbox}{#1}{#2}\end{lrbox}
}

\tikzstyle{tableau} = [y = -1cm, every node/.style={transform shape}]

\definecolor{gridColor}{RGB}{19,83,150}
\tikzstyle{dominoStyle} = [color=black, fill=white, rounded corners = .1cm, thick]
\tikzstyle{gridLine} = [color=gridColor, thick]
\tikzstyle{dominoText} = [font=\Large, midway]
\tikzstyle{cycleLine} = [color=green, thick, >->]
\tikzstyle{closedCycleLine} = [color=green, thick]
% \tikzstyle{fixedSquareStyle} = [pattern = crosshatch doats, pattern color=gridColor,  opacity=0.2]
\tikzstyle{fixedSquareStyle} = [color=gridColor,  opacity=0.07]
\tikzstyle{tileText} = [font=\large, midway]

\newcommand{\eps}{.06}
\newcommand{\teps}{\eps * 2}

% first entry is row, starting with 1, second entry is column, third is content
\newcommand{\filledSquare}[3]{\filldraw [dominoStyle] (#2 - 1 + \eps, #1 - 1 + \eps) rectangle + (1 - \teps, 1 -\teps) node [tileText] {$#3$};}
% The fourth entry shifts vertically
\newcommand{\filledSquareShift}[4]{\filldraw [dominoStyle] (#2 - 1 + #4 + \eps, #1 - 1 + \eps) rectangle + (1 - \teps, 1 -\teps) node [tileText] {$#3$};}

\newcommand{\horizontalDomino}[3]{\filldraw [dominoStyle] (#2 - 1 + \eps, #1 - 1 + \eps) rectangle + (2 - \teps, 1 -\teps) node [dominoText] {$#3$};}
\newcommand{\verticalDomino}[3]{\filldraw [dominoStyle] (#2 - 1 + \eps,  #1 - 1 + \eps) rectangle + (1 - \teps,2 -\teps) node [dominoText] {$#3$};}

\newcommand{\horizontalDominoShift}[4]{\filldraw [dominoStyle] (#2 - 1 + #4 + \eps, #1 - 1 + \eps) rectangle + (2 - \teps, 1 -\teps) node [dominoText] {$#3$};}
\newcommand{\verticalDominoShift}[4]{\filldraw [dominoStyle] (#2 - 1 + #4 + \eps,  #1 - 1 + \eps) rectangle + (1 - \teps,2 -\teps) node [dominoText] {$#3$};}

\newcommand{\zeroSquare}[2]{\filldraw [dominoStyle] (#2 - 1 + \eps, #1 - 1 + \eps) rectangle + (1 - \teps, 1 -\teps) node [dominoText] {$0$};}
\newcommand{\zeroSquareShift}[3]{\filldraw [dominoStyle] (#2 - 1 + #3 + \eps, #1 - 1 + \eps) rectangle + (1 - \teps, 1 -\teps) node [dominoText] {$0$};}


\newcommand{\emptyBox}[2]{\filldraw [dominoStyle] (#2 - 1 + \eps, #1 - 1 + \eps) rectangle + (2 - \teps, 2 -\teps);}
\newcommand{\signedBox}[3]{
\filldraw [opacity=0] (#2 - 1 + 1, #1 - 1) rectangle + (1, 2) node [dominoText,opacity=1] {$#3$};
\filldraw [dominoStyle, fill opacity = 0] (#2 - 1 + \eps, #1 - 1 + \eps) rectangle + (2 - \teps, 2 -\teps);
}

\newcommand{\emptyBoxShift}[3]{\filldraw [dominoStyle] (#2 - 1 + #3 + \eps, #1 - 1 + \eps) rectangle + (2 - \teps, 2 -\teps);}
\newcommand{\signedBoxShift}[4]{
\filldraw [opacity=0] (#2 - 1 + 1 + #4, #1 - 1) rectangle + (1, 2) node [dominoText,opacity=1] {$#3$};
\filldraw [dominoStyle, fill opacity = 0] (#2 - 1 + #4 + \eps, #1 - 1 + \eps) rectangle + (2 - \teps, 2 -\teps);
}

% These rows and columns are zero-based
\newcommand{\horizontalGridLine}[3]{\draw [gridLine] (#1, #2) -- + (#3,0);}
\newcommand{\verticalGridLine}[2]{\draw [gridLine] (#1, 0) -- + (0,#2);}
\newcommand{\fixedSquare}[2]{\filldraw [fixedSquareStyle] (#1,#2) rectangle +(1,1);}

% This will have #1 * 2 rows and #2 *2 columns
\newcommand{\gridLines}[2] {
  \pgfmathsetmacro{\verticalEnd}{2 * #1}
  \pgfmathsetmacro{\horizontalEnd}{2 * #2}
  \foreach \vertical in {0,...,#2} {
    \pgfmathsetmacro{\var} {2 * \vertical}
    \verticalGridLine{\var}{\verticalEnd}
  }
  \foreach \horizontal in {0,...,#1} {
    \pgfmathsetmacro{\var} {2 * \horizontal}
    \horizontalGridLine{0}{\var}{\horizontalEnd}
  }
}

% This will have #1 * 2 rows and #2 *2 columns
% The vertical lines will be shifted over #3 squares
\newcommand{\gridLinesShift}[3] {
  \pgfmathsetmacro{\verticalEnd}{2 * #1}
  \pgfmathsetmacro{\horizontalEnd}{2 * #2}
  \foreach \vertical in {0,...,#2} {
    \pgfmathsetmacro{\var} {2 * \vertical + #3}
    \verticalGridLine{\var}{\verticalEnd}
  }
  \foreach \horizontal in {0,...,#1} {
    \pgfmathsetmacro{\var} {2 * \horizontal}
    \horizontalGridLine{#3}{\var}{\horizontalEnd}
  }
}

\newcommand{\fixedSquaresStart}[4]{
  \foreach \row in {#1,...,#2} {
    \foreach \column in {#3,...,#4} {
      \pgfmathsetmacro{\var}{\row + \column}
      \ifodd \var
      \else
        \fixedSquare\column\row
      \fi
    }
  }
}

\newcommand{\fixedSquares}[2]{
  \foreach \row in {0,...,#1} {
    \foreach \column in {0,...,#2} {
      \pgfmathsetmacro{\var}{\row + \column}
      \ifodd \var
        \fixedSquare\column\row
      \fi
    }
  }
}

% This has #1 * 2 rows and #2 * 2 columns
\newcommand{\fixedSquaresForGrid}[2] {
  \pgfmathsetmacro{\rowParameter}{#1 * 2 - 1}
  \pgfmathsetmacro{\columnParameter}{#2 * 2 - 1}
  \fixedSquares{\rowParameter}{\columnParameter}
}

% This has #1 * 2 rows and #2 * 2 columns
% The vertical lines will be shifted over #3 squares
\newcommand{\fixedSquaresForGridShift}[3] {
  \pgfmathsetmacro{\rowParameter}{#1 * 2 - 1}
  \pgfmathsetmacro{\columnStart}{#3}
  \pgfmathsetmacro{\columnEnd}{#2 * 2 - 1 + #3}
  \fixedSquaresStart{0}{\rowParameter}{\columnStart}{\columnEnd}
}

\newcommand{\fixedSquaresStartAlt}[4]{
  \foreach \row in {#1,...,#2} {
    \foreach \column in {#3,...,#4} {
      \pgfmathsetmacro{\var}{\row + \column + 1}
      \ifodd \var
      \else
        \fixedSquare\column\row
      \fi
    }
  }
}

% This has #1 * 2 rows and #2 * 2 columns
% The vertical lines will be shifted over #3 squares
\newcommand{\fixedSquaresForGridShiftAlt}[3] {
  \pgfmathsetmacro{\rowParameter}{#1 * 2 - 1}
  \pgfmathsetmacro{\columnStart}{#3}
  \pgfmathsetmacro{\columnEnd}{#2 * 2 - 1 + #3}
  \fixedSquaresStartAlt{0}{\rowParameter}{\columnStart}{\columnEnd}
}


% This will have #1 rows and #2 columns
\newcommand{\typeAGridLines}[2] {
  \foreach \vertical in {0,...,#2} {
    \verticalGridLine{\vertical}{#1}
  }
  \foreach \horizontal in {0,...,#1} {
    \horizontalGridLine{0}{\horizontal}{#2}
  }
}

% This will have #1 rows and #2 columns
% The vertical lines will be shifted over #3 squares
\newcommand{\typeAGridLinesShift}[3] {
  \foreach \vertical in {0,...,#2} {
    \pgfmathsetmacro{\var} {\vertical + #3}
    \verticalGridLine{\var}{#1}
  }
  \foreach \horizontal in {0,...,#1} {
    \horizontalGridLine{#3}{\horizontal}{#2}
  }
}
